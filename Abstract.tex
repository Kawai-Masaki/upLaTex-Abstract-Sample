\documentclass[uplatex,a4paper,twocolumn,8pt,dvipdfmx]{jsarticle}

\usepackage[dvipdfmx]{graphicx}
\usepackage{diagbox} % 旧\usepackage{slashbox}
\usepackage{wrapfig}
\usepackage{setspace}
\usepackage{latexsym}
\usepackage{tabularx} % Table用
\usepackage{multicol} % セル結合用
\usepackage{multirow} % セル結合用
% \AtBeginDvi{\special{pdf:mapfile msmingoth.map}} % WindowsでMS明朝を使うとき用

\usepackage[top=15truemm,bottom=20truemm,left=16truemm,right=16truemm]{geometry}

\newcolumntype{C}[1]{>{\centering\arraybackslash}p{#1}}
\newcolumntype{L}[1]{>{\raggedright\arraybackslash}p{#1}}
\newcolumntype{R}[1]{>{\raggedleft\arraybackslash}p{#1}}

\title{uplatexを使った2段組文書作成}		%タイトル
\author{Haruka \thanks{国立高等専門学校 全知全能科}}
\date{2021年11月26日}
\begin{document} 
\maketitle
%タイトル

%本文
\section{はじめに}
最近は Over Leaf 等のオンラインの\LaTeX サービスがあるので,
それらを使うといいと思うが画像のコンパイル等が重くなるので
やっぱりローカルで環境を作ったほうがいいかもしれない.
upLaTex は Unicode が使えるので記号等の問題が少ないのがメリットかもしれない.


% \vspace{-0.5zh} % ちょっと行間を詰めた
\section{関連研究}
既存研究として,Vinyalsら\cite{Vinyals2015ShowAT}は,
畳み込みニューラルネットワーク(CNN)と
再起型ニューラルネットワーク(RNN)を用いた
エンコーダー・デコーダー構造により,
画像特徴とキャプションから,文の生成を学習する手法を提案している.

\section{提案手法}
Vinyalsらの提案するImage Captioningのエンコーダー・デコーダー構造を参考に,
人間を用いた人力キャプション生成を提案する(図\ref{fig:model}).
\begin{figure}[htbp]
    \centering
    \includegraphics[width=0.6\linewidth]{fig/human_captioning.jpg}
    \caption{絵を見て内容を言語にて回答}
    \label{fig:model}
\end{figure}

\section{生成実験}
友達と協力させる人力キャプション生成を実装する(図\ref{fig:ours_fr}).
既存手法としてVinyalsらの手法(NIC)と,人力キャプション生成モデル,
それに友人を導入したと友人協力モデルを比較した.
表\ref{tb:score}に示すように,人力キャプション生成モデルの方が
著しく高いスコアとなった.
また,友人協力モデルは最も高い評価となったものの,
人力キャプション生成モデルと比較すると差はわずかである.
% 表をトップに配置される
\begin{table}[tbp]
    \centering
    \small
    \caption{評価用データセットに対するキャプション生成の品質.
            1,2,3,4は$n$-gramsを使ったBleuスコア.
            数値が高いほどGround Truthとの類似度が高い.
            Ours+Fはモデルと友人(Friend)を相談させるネットワークとする.
            }
    \begin{tabular}{ L{1.0cm}  *{4}{C{0.5cm}}  C{0cm}  *{3}{C{0.55cm}} }
        %   \hline
        \hline
        & \multicolumn{4}{c}{Bleu} && \multicolumn{3}{c}{Bert} \\
        \cline{2-5}\cline{7-9}
                    &  1 &  2 &  3 &  4 &&  P &  R &  F1\\  \hline
        NIC\cite{Vinyals2015ShowAT} &  0.113 &  0.018 &  0.001 &  0.000 &&  0.856 &  0.878 &  0.867\\ 
        Ours                        &  {\bf 1.00} &  0.990 &  0.980 &  0.900 &&  0.990 &  0.990 &  0.990\\ 
        Ours+Fr                 &  {\bf 1.00} &  {\bf 1.00} &  {\bf 0.999} &  {\bf 0.990} &&  {\bf 0.999} &  {\bf 0.999} &  {\bf 0.999}\\ 
        \hline
    \end{tabular}
    \label{tb:score}
\end{table}

\begin{figure}[tbp]
    \centering
    \vspace{0.3cm}
    \includegraphics[clip, width=0.8\linewidth]{fig/kids_katawokumu.png}
    \caption{友人と協力モデル(Ours+Fr)の図}
    \label{fig:ours_fr}
    \vspace{-0.5cm}
\end{figure}

\section{まとめ・今後の課題}
完璧すぎてありません.

\renewcommand{\refname}{参考文献}

%% 参考文献
%% アルファベット順(五十音順)
%\bibliographystyle{jplain}
%% 参照順
\bibliographystyle{junsrt}
\bibliography{mybib}


\end{document}
